\section{Perturbation at branching time}
\label{app:perturb_branching_time}
The application  of the \ac{ams} and \ac{gktl} algorithms to a \emph{deterministic} dynamical system requires a perturbation at initial time of the
resampled trajectories. In the absence of such perturbation, the resampling would not yield new trajectories but exact copies of the original trajectories.
In the framework of Lattice Boltzmann simulations, the state of the system is described at a mesoscopic level by the particle densities $\{f_i(\mathbf{x},t)\}_{i=0\cdots8}$ (see appendix~\ref{app:lbm}).
%
Therefore, the perturbation of the solution at time $t_0$ applies directly to the $f_i$'s with 
\begin{equation}
f_{i}(\mathbf{x},t_0) \longrightarrow f_i(\mathbf{x},t_0) + \epsilon \sum_{n=1}^{N_s} \alpha_{n}f_{i}^{(n)}(\mathbf{x}),
\label{eq:perturb_pop}
\end{equation}
where the $\alpha_n$'s are random numbers uniformly picked in the interval $[0,1]$, $\epsilon$ is the relative amplitude of the perturbation and $\{f_i^{(n)}\}_{n=1 \cdots N_s}$ is a set of arbitrary snapshots of the flow. 
%
In other words, the perturbation is a  random linear combination of $N_s$ snapshots.
Eventually, the perturbed densities are rescaled so that the mass of the system is preserved.
%
In practice, we chose $\epsilon = 0.002$ and $N_s = 10$ to ensure that the perturbations remain sufficiently small, random and independent.
%
In order to check that this perturbation does not impact the statistics of drag fluctuations, a long simulation of duration $T_{tot} = 10^5 \tau_c$ has been performed with a periodic perturbation (with period $\tau_c / 2$) mimicking the perturbation of the clones in the algorithm (see section \ref{sec:rare_events_algorithms}).
%
Fig.~\ref{fig:pdf_drag_with_perturbation} shows that the statistics of the perturbed and unperturbed simulations are equivalent.

\begin{figure}
  \centering
\includegraphics[width=.7\linewidth]{fig23/fig23}
\caption{\EL{\ac{pdf}s of (normalised) drag fluctuations obtained from the perturbed (at branching points) and unperturbed numerical simulations; $\epsilon$ is the relative amplitude of the perturbation.}}
\label{fig:pdf_drag_with_perturbation}
\end{figure}
