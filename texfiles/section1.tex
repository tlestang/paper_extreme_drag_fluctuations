buffer-name
\section{Introduction}

% general comments on physical problem %
%
Turbulent flows are important in a variety of natural phenomena, industrial and civil applications.
Their characteristic feature is the spontaneous development of intense and sporadic motions associated with extreme internal forces \citep{lesieur_book,donzis_sreenivasan_2010,Yeung}.
``Extreme'' refers here to fluctuations that can deviate from the mean value by ${\cal{O}}(10)$ standard deviations.
In engineering, the nature of such extreme dynamical events and their statistics are of crucial interest to predict excessive mechanical efforts, which can threaten the structural integrity of embedded structures \citep{kanev2010}.
%

From the viewpoint of chaotic dynamical systems, turbulence in fluids is linked to non-linearity and strong departure from statistical equilibrium \citep{KRAICHNAN}.
The use of analytical perturbative methods in identifying resonant interactions (among degrees of freedom) responsible for extreme fluctuations is unsuccessful.
Alternatively, simulation offers a practical approach to gain physical insight into these events, quantifying their intensity and estimating their frequency of occurrence.
However, this requires very long simulations since these extreme events are rare. The computational cost for sampling a fluctuation of very small probability typically grows as the inverse of this probability~\citep{wouters2016rare}.
%
%
{Rare-event sampling refers to a large body of methods that aim at \emph{preferentially} exploring  regions of phase space corresponding to  events that would otherwise be accessed with a very low probability through a brute-force direct sampling.}
%
%	
% next two paragraphs may be too technical -- Find a way to summarize them in a few sentences accessible to a broad audience in fluid mechanics? Notion of action will be vague for many readers
%
In the present work, a computational study of extreme mechanical efforts acting on an immersed bluff body is conducted using both very long time-series (direct sampling) and rare-event sampling techniques.


%
In fluid turbulence, rare-event sampling has been approached mainly from the perspective of simplified dynamics such as the one-dimensional Burgers' equation with a stochastic forcing \citep{bec_burgers_2007}. In this case, dynamics can be sampled by using a Markov chain Monte-Carlo algorithm \citep{duben_monte_2008,mesterhazy2011anomalous,mesterhazy2013lattice} that provides a framework for rare-event sampling.
%
An alternative approach is based on instantons \citep{gurarie_instantons_1996,grafke2015instanton} and applies to stochastically driven systems in the limit of weak noise.
Instantons refer to the most probable trajectories in phase space that achieve a given rare event (in the limit of weak noise). Suitable numerical schemes can be used to evaluate instantons as well as the related probabilities of rare events \citep{chernykh_large_2001,grafke_instanton_2013,grigorio_instantons_2017,laurie2015computation,bouchet2014langevin}.
An example is the investigation of the physics of rogue waves~\citep{dematteis2018rogue,dematteis2019experimental}.
%
However, a drawback of the aforementioned approaches is their limitation to simple and stochastically driven dynamics.
%
%
%
Here, a more general approach is considered for complex, possibly deterministic, dynamical systems.
It is based on sampling algorithms relying on \emph{selection rules} applied to an ensemble of trajectories of the system. 
%
Even though such ideas date back to the early 1950s, they have received growing interest over the last twenty years with successful applications in various domains such as chemistry \citep{van_erp_elaborating_2005,escobedo_transition_2009,teo_adaptive_2016}, biophysics \citep{huber_weighted-ensemble_1996,zuckerman2017weighted,bolhuis2005kinetic}, nuclear physics \citep{louvin2017}, nonlinear dynamical systems \citep{tailleur_probing_2007} and communication networks simulation \citep{villen-altamirano_restart:_1994}.
More importantly, such algorithms have been shown to be useful for the study of rare events in simple deterministic dynamics~\citep{wouters2016rare}.
%
%
Certainly, an original contribution of our study is to test the application of rare-event sampling algorithms in the context of far-from-equilibrium dynamics with an irreducible very large number of degrees of freedom. 
%
Two algorithms that are a priori suitable for such dynamics are considered, namely, the Adaptive Multilevel Splitting (AMS) algorithm \citep{cerou_adaptive_2007} and the Giardina-Kurchan-Tailleur-Lecomte (GKTL) algorithm \citep{giardina_direct_2006}.
%
Another problem related to rare events is the estimation of parameters or statistics from a limited number of samples.  Several authors have proposed original strategies  \citep{Mohamad-ws,Blonigan-ws} such as sequential sampling. These approaches and rare event algorithms could also be considered for applications in wave-structure problems, ship motion and load acting on offshore platforms \citep{belenky-ws}.
%
The paper is organized in two parts. The first part highlights the phenomenology of extreme fluctuations of the drag force acting on a square placed in a two-dimensional turbulent channel flow. This study is based on the simulation of the flow over a very long duration, made possible by the relative simplicity of the flow.
%
The motivation for this study is twofold.
Firstly, it provides a detailed description of the statistics and dynamics related to extreme drag fluctuations. This analysis is informative from the  viewpoint of fluid mechanics and, to the best of our knowledge, has never been reported before.
Secondly, it yields reference results that are required to validate the outputs of rare-event algorithms and to evaluate the possible computational gain obtained from them. This assessment is developed in the second part of the paper.

% Annonce du plan
The flow set-up is introduced and the dynamics related to typical
drag fluctuations is described in section~\ref{sec:test_flow}.
The statistical properties of the drag are then discussed.
In section~\ref{sec:direct_sampling}, the phenomenology of  extreme drag fluctuations is investigated based on direct sampling.
Both the instantaneous drag and time-averaged drag are considered.
It is found in particular that sampled extreme fluctuations of the instantaneous drag result from very similar dynamics. 
Section~\ref{sec:rare_events_algorithms} examines the applicability of both the \ac{ams} and \ac{gktl} algorithms to the simulation of extreme drag fluctuations in the same flow configuration.
In subsection~\ref{sec:ams}, we show that the use of the \ac{ams} algorithm is not successful, or at least not straightforwardly. This difficulty is put in perspective with the phenomenology of extreme drag fluctuations developed in  section \ref{sec:direct_sampling}.
Subsection~\ref{sec:gktl} presents the computation of extremes of the time-averaged drag by using the \ac{gktl} algorithm.
This latter allows for an exceptional reduction of the computational cost needed to simulate trajectories corresponding to extreme values of the time-averaged drag.
As a specific successful application, the \ac{gktl} algorithm is used to compute the return times of extreme fluctuations of the time-averaged drag acting on the immersed obstacle.
Perspectives and conclusion end this work.