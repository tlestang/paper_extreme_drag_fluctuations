\section{Introduction}

% general comments on physical problem %
%
A characteristic feature of turbulent flows is the spontaneous development of intense and sporadic motions associated with extreme internal forces \citep{lesieur_book,donzis_sreenivasan_2010,Yeung}.
By ``extreme'' we refer to fluctuations that can deviate from the mean value by ${\cal{O}}(10)$ standard deviations.
In engineering, better understanding of the nature of such extreme dynamical events and their statistics is crucial for the control of excessive mechanical efforts, for instance extreme wind loads on tall structures such as buildings of wind turbines~\cite{kanev2010}.
%

From the viewpoint of chaotic dynamical systems, turbulence in fluids is linked to non-linearity and strong departure from statistical equilibrium \citep{KRAICHNAN}.
However, the use of analytical perturbative methods in identifying resonant interactions among degrees of freedom responsible for extreme fluctuations is unsuccessful.
Alternatively, numerical simulation offers a convenient approach to gain physical insight into these events, as well as quantifying their intensity and estimating their frequency of occurrence.
The main issue with this approach is however the very large durations over which the flows must be simulated in order to sample extreme events, since these are rare events.
The majority of flows relevant to industrial or environmental problems are three-dimensional and involve complex shapes as well as large Reynolds numbers.
In this context, tremendous computational requirements make the sole accurate simulation of the flow a challenge, and its computation over very long durations is impractical.

% general def on rare-event sampling %
%
More generally, the computational cost for sampling a fluctuation of very small probability grows like the inverse of this probability~\cite{wouters2016rare}, and simulating rare events through direct numerical simulation is therefore not a scalable approach.
This issue is addressed by rare-event sampling algorithms, referring to a large body of methods that aim at exploring preferentially regions of phase space corresponding to rare events, that would otherwise be accessed with a very low probability through a brute-force direct sampling.
%
Even though such ideas date back to the early 1950s, they have received ever-growing interest over the the last twenty years with successful applications in various domains such as chemistry \citep{van_erp_elaborating_2005,escobedo_transition_2009,teo_adaptive_2016}, biophysics \citep{huber_weighted-ensemble_1996,zuckerman2017weighted,bolhuis2005kinetic}, nuclear physics \citep{louvin2017}, nonlinear dynamical systems \citep{tailleur_probing_2007} and communication networks simulation \citep{villen-altamirano_restart:_1994}.
More importantly, these type of algorithms have been shown to be useful for the study of rare events in simple deterministic dynamics~\citep{wouters2016rare}.

In the present work, a computational study of extreme mechanical efforts acting on an immersed bluff body is conducted.
We deliberately focus on a simple two-dimensional flow in order to allow for its simulation over very large durations and sampling of extreme events.
Furthermore, we consider the application of rare-event sampling algorithms to reduce the computational cost associated to the sampling of these events.
The motivation for this work is twofold.
First, we provide a detailed description of the statistics and dynamics of rare events related to extreme
drag forces acting on a square placed in a two-dimensional channel flow.
This cannot be achieved based on more complex (and realistic) systems.
Although our analysis is based on a simplified geometry, it is important to fluid dynamics because it paves the way to a better understanding of this type of events.
To the best of the authors' knowledge, such analysis has never been reported before.
Second, we assess the relevance of rare-event sampling approaches for this problem.
In fluid turbulence, rare-event sampling has been approached mainly from the perspective of simplified dynamics such as the one-dimensional Burgers' equation with a stochastic forcing \citep{bec_burgers_2007}. In this case, dynamics can be sampled by using a Markov chain Monte-Carlo algorithm \citep{duben_monte_2008,mesterhazy2011anomalous,mesterhazy2013lattice} that provides a framework for rare-event sampling.
%
An alternative approach is based on instantons \citep{gurarie_instantons_1996,grafke2015instanton} and applies to stochastically driven systems in the limit of weak noise.
Instantons refer to the most probable trajectories in phase space that achieve a given rare event (in the limit of weak noise). Suitable numerical schemes can be used to evaluate instantons as well as the related probabilities of rare events \citep{chernykh_large_2001,grafke_instanton_2013,grigorio_instantons_2017,laurie2015computation,bouchet2014langevin}.
An example is the investigation of the physics of rogue waves~\citep{dematteis2018rogue,dematteis2019experimental}.
%
A drawback of the aforementioned approaches is their limitation to simple and stochastically driven dynamics.
%
%
In this paper, a more general approach is considered for complex, possibly deterministic, dynamical systems.
It is based on sampling algorithms relying on \emph{selection rules} applied to an ensemble of trajectories, and is designed to sample rare events of some observable with a higher frequency.
%
%
An original contribution of the present work is certainly to test the application of rare-event sampling algorithms in the context of far-from-equilibrium dynamics with an irreducible very large number of degrees of freedom.
%
Two different algorithms suitable for out-of-equilibrium dynamics are considered and compared. Namely, the \ac{ams} algorithm and the \ac{gktl} algorithm.

% Annonce du plan
This paper is organised in two main parts.
Section~\ref{sec:direct_sampling} describes the phenomenology of extreme drag events based on the simulation of the flow over a very long duration.
Section~\ref{sec:rare_events_algorithms} presents the application of two rare-events algorithms to the same
problem.
The flow set-up is introduced in section~\ref{sec:test_flow} and the statistical properties of the drag are discussed.
In section~\ref{sec:direct_sampling}, the phenomenology of  extreme drag fluctuations is investigated based on a direct sampling approach.
Both the instantaneous drag and time-averaged drag are considered.
It is found that sampled extreme events for the instantaneous drag share very similar dynamics.
Furthermore, extreme fluctuations for the time-averaged drag can be connected to the statistics of the instantaneous drag.
Section~\ref{sec:rare_events_algorithms} reports the applicability of both the \ac{ams} and \ac{gktl} algorithms to the numerical simulation of extreme drag fluctuations, by using the same flow configuration.
In section~\ref{sec:ams}, we show that the use of the \ac{ams} algorithm is not successful, or at least not straightforwardly.
This difficulty is put in perspective with the phenomenology developed in the previous sections.
Section~\ref{sec:gktl} presents the computation of extremes of the time-averaged drag, using the \ac{gktl} algorithm.
This latter allows for an exceptional reduction of the computational cost required to simulate trajectories corresponding to extreme time-averaged drag values.
As a specific successful application, the \ac{gktl} algorithm is used to compute the return times of extreme fluctuations of the time-averaged drag acting on the immersed obstacle.
Perspectives and conclusion end this work.

%%% Local Variables:
%%% mode: latex
%%% TeX-master: "draft_p2_jfm"
%%% End: