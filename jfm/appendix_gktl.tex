\section{The \ac{gktl} algorithm}
\label{app:gktl}
The \ac{gktl} algorithm is based on the simulation of an ensemble of $N$ trajectories $\left\{\mathbf{x}_{n}(t)\right\}_{0\leq t \leq T_a}$ with $ n =1 \cdots N$ starting from independent random initial conditions.
%
Let us consider a real-valued observable of interest $A(\mathbf{x}(t))$, {\emph{e.g.} the drag $f_d(t)$}, and introduce a cloning period $\tau$.
%
At time instants $t_{i}=i\tau$ with $i=1,~2,~...,~T_{a}/\tau$ ($T_{a}$ is a multiple of $\tau$) a weight $W_{n}^{i}$ is assigned to each trajectory. This weight is defined ($t_0=0$) by
%
\begin{equation}
W_{n}^{i}=\frac{e^{k\intop_{t_{i-1}}^{t_{i}}A(\mathbf{x}_{n}(t))dt}}{R_{i}}\quad \mbox{with the normalisation factor} \quad R_{i}=\frac{1}{N}\sum_{n=1}^{N}e^{k\int_{t_{i-1}}^{t_{i}}A(\mathbf{x}_{n}(t))dt}
\label{eq:Weight}
\end{equation}
so that $\sum_{n=1}^N W_n^i = N$.
%
%
{The weights $\{W_{n}^{i}\}_{n=1\cdots N}$ determine how many copies of each trajectory are made at time $t=t_i$. The parameter $k$ characterizes the amplitude of the statistical bias involved in the algorithm (see Fig.~\ref{fig:IS_GKTL}). For more information about the practical implementation of the algorithm, the interested reader can refer to~\cite{brewer2018efficient, lestang:tel-01974316}}.
The application of this re-sampling at each step $t_i$ eventually leads to a biased sampling in the trajectory space; the trajectories corresponding to extreme values of $\int_{0}^{T_a}A(\mathbf{x}_{n}(t))dt$ have a higher probability.
%
The sampled biased distribution writes
%
\begin{align}
\mathbb{P}_{k}\left(\left\{ \mathbf{X}(t)\right\} _{0\leq t\leq T_{a}}=\left\{ \mathbf{x}(t)\right\} _{0\leq t\leq T_{a}}\right) &\underset{N\rightarrow\infty}{\sim} \frac{e^{k\int_{0}^{T_{a}}A(\mathbf{x}(t))dt}}{Z(k,T_a)}\mathbb{\mathbb{P}}_{0}\left(\left\{ \mathbf{X}(t)\right\} _{0\leq t\leq T_{a}}=\left\{ \mathbf{x}(t)\right\} _{0\leq t\leq T_{a}}\right),
\label{eq:Biased_Path_Approximation}
\end{align}
where
$\mathbb{P}_{0}\left(\left\{ \mathbf{X}(t)\right\} _{0\leq t\leq T_{a}} = \left\{ \mathbf{x}(t)\right\} _{0\leq t\leq T_{a}}\right)$
refers formally to the probability of observing the trajectory
$\left\{ \mathbf{x}(t)\right\} _{0\leq t\leq T_{a}}$.
The normalisation factor is given by $Z(k,T_a)=\prod_{i=1}^{T_a/\tau}R_i$.
%
One can mention that
\begin{equation}
\label{eq:mean_field}
Z(k,T_a) \underset{N\to \infty}{\sim} \mathbb{E}_0\left[e^{k\int_{0}^{T_{a}}A(\mathbf{X}(t))dt}\right],
\end{equation}
with $\mathbb{E}_{0}$ being the expectation value with respect to the
distribution $\mathbb{P}_{0}$.
This result relies on the \textit{mean-field approximation}
\begin{equation}
R_{i}=\frac{1}{N}\sum_{n=1}^{N}e^{k\int_{t_{i-1}}^{t_{i}}A(\mathbf{X}_{n}(t))dt}\underset{N\rightarrow\infty}{\sim} Z(k,t_i)= \mathbb{E}_{i}\left[e^{k\int_{t_{i-1}}^{t_{i}}A(\mathbf{X}(t))dt}\right],
\label{eq:Mean_Field_Approximation}
\end{equation}
where $\mathbb{E}_{i}[.]$ denotes the expectation value with respect to the biased distribution $\mathbb{P}_k^{(i)}$ obtained after $i$ cloning steps.
The typical relative error related to this approximation can be shown to be of order $1/\sqrt{N}$ for a family of rare-event algorithms including the \ac{gktl} algorithm~\citep{DelMoralBook,DelMoral2013}.
%
Rejected trajectories are discarded from the statistics.
Eventually, an effective ensemble of $N$ trajectories of duration $T_{a}$ is obtained, distributed according to $\mathbb{P}_{k}$.

A key feature of the \ac{gktl} algorithm is that the sampling procedure does not involve any alteration of the dynamics.
All resampled trajectories are solutions of the original dynamical system.
Nevertheless, it should be noted that a small random perturbation is introduced in the cloning procedure to force clones from the same trajectory to separate, \emph{i.e.} artificial randomness is introduced so that the cloning procedure is effective for deterministic dynamics.
It was checked that this perturbation does not affect the statistics of the sampled trajectories, see appendix~\ref{app:perturb_branching_time}.

Eventually, the sampled trajectories obtained with the \ac{gktl} algorithm can be used to compute the statistical properties of any observable with respect to the distribution $\mathbb{P}_{0}$ from the distribution $\mathbb{P}_{k}$ by using Eq.~\eqref{eq:Biased_Path_Approximation}.

%%% Local Variables:
%%% mode: latex
%%% TeX-master: "draft_p2_jfm"
%%% End: