\section{The \acl{lbm}}
\label{app:lbm}

% details about LBM
In the LB method, the fluid is viewed as a population of particles that collide, redistribute and propagate along the different links of a discrete lattice (see \cite{kruger_lattice_2017} for a comprehensive introduction).
In our two-dimensional situation, the so-called D2Q9 lattice with only nine possible velocities $\{\mathbf{c_i}\}_{i=0...8}$ at each node has been adopted (see  Fig.~\ref{fig:D2Q9}).
Locally, the macroscopic flow variables (per unit volume) are recovered by summing over the densities of particles $\{f_i\}_{i=0...8}$ moving with the different velocities, i.e.
\[
\rho(\mathbf{x},t) = \sum_i f_i(\mathbf{x},t) \quad \mathrm{and}\quad \rho(\mathbf{x},t) \mathbf u(\mathbf{x},t) = \sum_i f_i(\mathbf{x},t) \mathbf{c_i}
\]
for the mass density and the fluid momentum respectively. The assumption of weak compressibility (for an ideal gas) is made so that the pressure is directly proportional to the mass density: $p = c_s^2 \rho$ where $c_s$ is interpreted as a speed of sound.

\begin{figure}
	\centering
	\includegraphics[width=0.3\linewidth]{D2Q9/D2Q9}
	\caption{Sketch of the D2Q9 lattice. Particles move exactly from a lattice node towards one of its nine neighbours (including the node itself) during one time step. By definition, the lattice spacing is related to the time step by $\Delta x/ \Delta t = \sqrt{3} c_s$ where $c_s$ is interpreted as a speed of sound.}
	\label{fig:D2Q9}
\end{figure}


% algo
%
The complexity of the flow emerges from the repeated application of simple rules of streaming and collision. The \ac{lbm} advances the local densities of particles $f_i(\mathbf{x},t)$ moving with velocities $\mathbf{c}_i$  in a two-step procedure. Namely, an \emph{exact} streaming step
\begin{equation}
  \label{eq:lbe}
  f_i(\mathbf{x}+\mathbf{c}_i \Delta t, t + \Delta t) = f_i^{\mathrm{out}}(\mathbf{x},t)
\end{equation}
during which particles move with their own velocity to a neighbouring node, and an instantaneous collision step
\[
f_i^{\mathrm{out}}(\mathbf{x},t) = -\frac 1 {\tau_\nu} \left(f_i(\mathbf{x},t) - f_i^\mathrm{eq}(\mathbf{x},t) \right)
\]
which achieves a relaxation of local densities towards an absolute equilibrium (at the macroscopic level). The time-scale $\tau_\nu$ (in lattice unit) is related to the kinematic viscosity of the fluid by
\[
\nu = \left( {\tau_\nu} - \frac 1 2 \right) c_s^2 ~\Delta t
\]
This simplification of the collision kernel is known as the BGK approximation in the kinetic theory of gas \citep{BGK}.
%
The equilibrium function is given by
\begin{equation}
  \label{eq:lbe_eq}
  f_i^\mathrm{eq}(\mathbf{x},t) = w_i  \rho(\mathbf{x},t) \left( 1 + \frac{\mathrm u(\mathbf{x},t) \cdot \mathbf{c_i}}{c_s^2} +
    \frac{u_\alpha(\mathbf{x},t) u_\beta(\mathbf{x},t)({c_i}_\alpha {c_i}_\beta - c_s^2 \delta_{\alpha\beta})}{2 c_s^4} \right),
\end{equation}
with the weight factors $w_0=4/9,~w_{1...4} = 1/9$ and $w_{5...8}=1/36$ for the D2Q9 lattice.
This discrete Lattice Boltzmann scheme is second-order accurate in $\Delta x $ and compliant to the weakly-compressible Navier-Stokes equations with a third-order error in $\mathrm{Ma}=|\mathbf{u}|/c_s$ as the lattice spacing vanishes, i.e. $\Delta x \to 0$ \citep{succi_book}.

As mentioned before, the pressure is directly accessible from the mass density: $p = \rho c_s^2$. The viscous stress is also obtained easily from the densities of particles by
\[
\tau^\mathrm{visc.}_{\alpha \beta} = -\frac{\nu}{\tau_\nu ~ c_s^2 \Delta t} \sum_i  {c_i}_\alpha {c_i}_\beta (f_i - f_i^\mathrm{eq})
\]
so that the total stress expresses as
\begin{equation}\label{eq:def_stress}
\tau_{\alpha \beta} = -  c_s^2 \sum_i f_i ~ \delta_{\alpha\beta}  - \frac{\nu}{\tau_\nu ~ c_s^2 \Delta t} \sum_i  {c_i}_\alpha {c_i}_\beta (f_i - f_i^\mathrm{eq})
\end{equation}
Finally, let us mention that in the present context of turbulent flows, the single-relaxation-time BGK collision has been replaced by a multi-relaxation-time procedure based on central moments with an improved stability \citep{De_Rosis_2016}.

%%% Local Variables:
%%% mode: latex
%%% TeX-master: "draft_p2_jfm"
%%% End: